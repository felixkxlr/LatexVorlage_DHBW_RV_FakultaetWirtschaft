\documentclass[a4paper,12pt]{article}

% ------------------------------------------------------------------------------
% 1. KONFIGURATION & SPRACH-LOGIK (Muss zuerst geladen werden)
% ------------------------------------------------------------------------------
\def\myType{0}
%% 0=Projektarbeit   %%
%% 1=Bachelorarbeit  %%
%% 2=sonstige Arbeiten   %%

\def\myTopic{Titel}
\def\mySubTopic{Subtitel}

\def\myStudyProgram{}         % Studiengang
\def\myNeedForFieldOfStudy{0} % Gibt an ob eine Studienrichtung benötigt wird
%% 0=keine Studienrichtung %%
%% 1=Studienrichtung %%
\def\myFieldOfStudy{}         % Studiengangsrichtung

\def\myAutor{}
\def\myCourse{}
\def\myProf{}
\def\myCompany{}
\def\myCompanion{}            % Begleitperson des dualen Partners
\def\myEndDate{}              % Abgabefrist


%% Projektarbeit %%
\def\myProjNumber{1}          % [1|2]

%% Bachelorarbeit %%
\def\myTypeOfBachelor{}       % [Arts|Science]

%% sonstige Arbeiten %%
\def\myTypeOfWork{}           % [Seminararbeit|Projektdokumeantation|...]
\def\myModule{}               % Modulname
\def\myNames{0}               % Gibt an ob mehrere Personen an der Arbeit mitwirken
%% 0=Einzahl %%
%% 1=Mehrzahl %%

%% Konfiguration der Verzeichnisse %%
% für alle folgenden Konfigurationen gilt: 0=kein Verzeichnis | 1=Verzeichnis %

\def\myAbrevationDirectory{1}   % Abkürzungsverzeichnis
\def\myIllustrationDirectory{1} % Abbildungsverzeichnis
\def\myTableDirectory{1}        % Tabellenverzeichnis
\def\myListingDirectory{1}      % Listingverzeichnis
\def\myFormularDirectory{1}     % Formelverzeichnis
\def\mySymbolDirectory{1}      % Symbolverzeichnis

%% Konfiguration Selbständigkeitserklärung und Sperrvermerk %%
\def\myDecOfIndependence{1}
%% 0=keine Selbständigkeitserklärung   %%
%% 1=Selbständigkeitserklärung    %%

\def\myBlockingNote{0}
%% 0=kein Sperrvermerk   %%
%% 1=Sperrvermerk    %%

% Ort und Datum der Unterschrift auf Selbständigkeitserklärung und Sperrvermerk
\def\myDate{}
\def\myPlace{}

% Sprachauswahl: 'ngerman' oder 'english'
\newcommand{\myLanguage}{ngerman}           % Lädt \myLanguage und andere Einstellungen
\usepackage{ifthen}             % Ermöglicht If-Abfragen für die Sprache

% Sprache laden (basierend auf \myLanguage aus config.tex)
% WICHTIG: inputenc sollte davor stehen
\usepackage[utf8]{inputenc}
\usepackage[\myLanguage]{babel} 

% Sprach-Definitionen laden (Deutsch oder Englisch)
\ifthenelse{\equal{\myLanguage}{ngerman}}
    {% Datei: main.tex
\newcommand{\langAnhang}{Anhang}       
\newcommand{\langFigureName}{Abb.}     
\newcommand{\langTableName}{Tab.}      

% Datei: 10_titel_bachelor.tex
\newcommand{\langBachelorThesis}{Bachelorarbeit}
\newcommand{\langForDegree}{für die\\Prüfung zum Bachelor of}
\newcommand{\langFaculty}{an der Fakultät Wirtschaft}
\newcommand{\langStudyProgram}{im Studiengang}
\newcommand{\langFieldOfStudy}{in der Fachrichtung}
\newcommand{\langAt}{an der}
\newcommand{\langAutor}{Autor/-in}
\newcommand{\langKurs}{Kurs}
\newcommand{\langBetreuer}{DHBW-Betreuer/-in}
\newcommand{\langFirma}{Dualer Partner}
\newcommand{\langBegleiter}{Begleiter/-in Dualer Partner}
\newcommand{\langAbgabefrist}{Abgabefrist}

% Datei: 10_titel_projekt_doku.tex
\newcommand{\langAutoren}{Autoren/Autorinnen}
\newcommand{\langModul}{Modul}

% Datei: 10_titel_projekt.tex
\newcommand{\langProjektarbeit}{Projektarbeit}

% Datei: 13_abkuerzungsverzeichnis.tex
\newcommand{\langAbkuerzungsverzeichnis}{Abkürzungsverzeichnis}

% Datei: 14_Formelverzeichnis.tex
\newcommand{\langFormelverzeichnis}{Formelverzeichnis}
\newcommand{\langFormelPrefix}{FV.}

% Datei: 17_listingverzeichnis.tex
\newcommand{\langListingsverzeichnis}{Listingverzeichnis}
\newcommand{\langListingName}{Listing}

% Datei: 18_literaturverzeichnis.tex
\newcommand{\langLiteraturverzeichnis}{Quellen- und Literaturverzeichnis}

% Datei: 19_symbolverzeichnis.tex
\newcommand{\langSymbolverzeichnis}{Symbolverzeichnis}
} % Lädt deutsche Texte & Anpassungen
    {% Datei: main.tex
\newcommand{\langAnhang}{Appendix}     
\newcommand{\langFigureName}{Fig.}     
\newcommand{\langTableName}{Tab.}     

% Datei: 10_titel_bachelor.tex
\newcommand{\langBachelorThesis}{Bachelor Thesis}
\newcommand{\langForDegree}{for the\\examination for the degree of Bachelor of}
\newcommand{\langFaculty}{at the School of Business}
\newcommand{\langStudyProgram}{in the study program}
\newcommand{\langFieldOfStudy}{in the field of study}
\newcommand{\langAt}{at the}
\newcommand{\langAutor}{Author}
\newcommand{\langKurs}{Course}
\newcommand{\langBetreuer}{DHBW Supervisor}
\newcommand{\langFirma}{Dual Partner}
\newcommand{\langBegleiter}{Company Supervisor}
\newcommand{\langAbgabefrist}{Submission Date}

% Datei: 10_titel_projekt_doku.tex
\newcommand{\langAutoren}{Authors}
\newcommand{\langModul}{Module}
\newcommand{\langDozent}{Lecturer}


% Datei: 10_titel_projekt.tex
\newcommand{\langProjektarbeit}{Project Work}

% Datei: 13_abkuerzungsverzeichnis.tex
\newcommand{\langAbkuerzungsverzeichnis}{List of Abbreviations}

% Datei: 14_Formelverzeichnis.tex
\newcommand{\langFormelverzeichnis}{List of Formulas}
\newcommand{\langFormelPrefix}{LF.}

% Datei: 17_listingverzeichnis.tex
\newcommand{\langListingsverzeichnis}{List of Listings}
\newcommand{\langListingName}{Listing}

% Datei: 18_literaturverzeichnis.tex
\newcommand{\langLiteraturverzeichnis}{Bibliography}

% Datei: 19_symbolverzeichnis.tex
\newcommand{\langSymbolverzeichnis}{List of Symbols}} % Lädt englische Texte & Anpassungen

% ------------------------------------------------------------------------------
% 2. LAYOUT & STANDARD-PAKETE
% ------------------------------------------------------------------------------
\usepackage[left=2.5cm,right=2.5cm,top=2.5cm,bottom=2cm,includehead]{geometry}
\usepackage[hyperfootnotes=false, pdfborder={0 0 1}]{hyperref} 
\usepackage[nottoc]{tocbibind}
\usepackage{makeidx}
\usepackage[intoc]{nomencl}
\usepackage{fancyhdr}
\usepackage{pdfpages}
\usepackage{amsmath}
\usepackage[labelfont=bf, aboveskip=1mm, font={small, singlespacing}]{caption}
\usepackage{setspace}
\usepackage[bottom,multiple,hang,marginal]{footmisc}
\usepackage{wrapfig}
\usepackage{nomencl} 
\usepackage{chngcntr}
\usepackage{import}
\usepackage{graphicx}
\usepackage{tabularx}
\usepackage{longtable}
\usepackage{color}
\usepackage{enumitem}
\usepackage{listings}
\usepackage{zref}
\usepackage{wasysym}
\usepackage{amssymb}
\usepackage{xspace}
\usepackage{multicol}
\usepackage{booktabs}
\usepackage{siunitx}
\RequirePackage{multirow}
\RequirePackage{mwe}
\RequirePackage{etoolbox}
\usepackage{float}
\usepackage{adjustbox}
\usepackage{comment}
\usepackage{subcaption}
\usepackage{hanging}
\usepackage{acronym}
\usepackage{csquotes}

\pdfminorversion=6

%%%%%%%%%%%%%%%%%%%%%%%% Eigene Farbwerte definieren %%%%%%%%%%%%%%%%%%%%%%%%
\definecolor{boxgray}{gray}{0.9}     
\definecolor{commentgray}{gray}{0.5} 
\definecolor{darkgreen}{rgb}{0,.5,0} 

%%%%%%%%%%%%%%%%%%%%%%%% Eigene Kommandos definieren %%%%%%%%%%%%%%%%%%%%%%%%
\newcommand{\gqq}[1]{\glqq #1\grqq}

\providecommand*{\footref}[1]{
	\begingroup
		\unrestored@protected@xdef\@thefnmark{\ref{#1}}
	\endgroup
\@footnotemark}

\newcommand{\mypageref}[1]{\ref{#1} \nameref{#1} auf Seite \pageref{#1}}

\newcommand{\myboxquote}[1]{
	\begin{quotation}
		\colorbox{boxgray}{\parbox{0.78\textwidth}{#1}}
	\end{quotation}
	\vspace*{1mm}
}

% Definition von \vgl{#1}{#2} (APA-konform)
\newcommand{\vgl}[2][]{\parencite[vgl.][#1]{#2}}


\makeatletter
\zref@newprop*{appsec}{}
\zref@addprop{main}{appsec}

\def\applabel#1#2{%
	\zref@setcurrent{appsec}{#2}%
	\zref@wrapper@immediate{\zref@label{#1}}%
}

\def\appref#1{%
	\hyperref[#1]{\zref@extract{#1}{appsec}}%
}
\makeatother

% Definition von \appsection: Nutzt jetzt \langAnhang Variable
\newcommand{\theappsection}[1]{\langAnhang \Alph{section}:~\protect #1}
\newcommand{\appsection}[2]{
	\addtocounter{section}{1}
	\phantomsection
	\addcontentsline{toc}{section}{\theappsection{#1}}
	\markboth{\theappsection{#1}}{}

	\section*{\theappsection{#1}}
	\applabel{#2}{\langAnhang \Alph{section}}
	\label{#2}
}

% Nummerierung der Formeln
\renewcommand{\theequation}{\arabic{equation}}

%%%%%%%%%%%%% Index, Abkürzungsverzeichnis %%%%%%%%%%%%
\makeindex
\makenomenclature

% Zitation (APA 7 mit DHBW Sonderregel maxcitenames=3)
\usepackage[style=apa, backend=biber, language=ngerman, maxcitenames=3, mincitenames=1]{biblatex}
\addbibresource{literatur.bib}

%%%%%%%%%%%%%%%%%%%%%%%%%%%%%%% PDF-Optionen %%%%%%%%%%%%%%%%%%%%%%%%%%%%%%%%
\hypersetup{
	bookmarksopen=false,
	bookmarksnumbered=true,
	bookmarksopenlevel=0,
	pdftitle=\myTopic,
	pdfsubject=\myTopic,
	pdfauthor=\myAutor,
	pdfborder={0 0 1}, 
	pdfstartview=Fit,
	pdfpagelayout=SinglePage
}

%%%%%%%%%%%%%%%%%%%%%%%%%%%% Kopf- und Fußzeile %%%%%%%%%%%%%%%%%%%%%%%%%%%%%
\pagestyle{fancy}
\fancyhf{} 
\fancyfoot[R]{\thepage}
\renewcommand{\headrulewidth}{0.5pt} 
\renewcommand{\footrulewidth}{0pt} 

%%%%%%%%%%%%%%%%%%%%%%%%% Layout und Beschriftungen %%%%%%%%%%%%%%%%%%%%%%%%%
\onehalfspacing
\setlist{noitemsep}

\setlength{\parskip}{6pt}
\setlength{\parindent}{0pt}

% Dynamische Anpassung der Bild/Tabellen-Unterschriften basierend auf der Sprache
\addto\captionsngerman{
  \renewcommand{\figurename}{\langFigureName}
  \renewcommand{\tablename}{\langTableName}
}
\addto\captionsenglish{
  \renewcommand{\figurename}{\langFigureName}
  \renewcommand{\tablename}{\langTableName}
}

\numberwithin{table}{section}
\numberwithin{figure}{section}
\renewcommand{\thetable}{\arabic{section}.\arabic{table}}
\renewcommand{\thefigure}{\arabic{section}.\arabic{figure}}
\renewcommand{\thefootnote}{\arabic{footnote}}
\renewcommand{\multfootsep}{, }

%%%%%%%%%%%%%%%%%%%%%%%%%%%%%%% Listingstyle %%%%%%%%%%%%%%%%%%%%%%%%%%%%%%%%
\lstset{
	basicstyle=\ttfamily\scriptsize,
	commentstyle=\color{commentgray}\textit,
	showstringspaces=false,
	stringstyle=\color{darkgreen},
	keywordstyle=\color{blue},
	numbers=left,
	numberstyle=\tiny,
	stepnumber=1,
	numbersep=15pt,
	tabsize=2,
	fontadjust=true,
	frame=single,
	backgroundcolor=\color{boxgray},
	captionpos=b,
	linewidth=0.94\linewidth,
	xleftmargin=0.1\linewidth,
	breaklines=true,
	aboveskip=16pt
}

%%%%%%%%%%%%%%%%%%%%%%%%%%%%%%%%%%%%%%%%%%%%%%%%%%%%%%%%%%%%%%%%%%%%%%%%%%%%%
%%                         DOKUMENT BEGINN                                 %%
%%%%%%%%%%%%%%%%%%%%%%%%%%%%%%%%%%%%%%%%%%%%%%%%%%%%%%%%%%%%%%%%%%%%%%%%%%%%%

\begin{document}
	\pagenumbering{Roman}
	
	% Titelblatt (Dateien bleiben gleich, Variablen werden innen genutzt)
	\ifcase\myType
            \begin{titlepage}
	\begin{center}
		\vspace*{1cm}
		\LARGE\bf\myTopic\\
		\Large\rm\mySubTopic\\
		\vspace*{2cm}
		\bf \myProjNumber.~\langProjektarbeit\\
		\vspace*{0.5cm}\singlespacing
		\normalsize\rm
		\ifcase\myNeedForFieldOfStudy
            \langFaculty\\
            \langStudyProgram \myStudyProgram \\
        \or 
            \langFaculty\\
            \langStudyProgram \myStudyProgram \\
            \langFieldOfStudy \myFieldOfStudy\\
        \else
        \fi
		\vspace*{0.5cm}\singlespacing
		\langAt\\
		DHBW Ravensburg
		\vfill
	\end{center}
	\begin{tabular}{lll}
		\langAutor: &\myAutor\\
            \langKurs: &\myCourse\\
            \langBetreuer: &\myProf\\
		\langFirma: &\myCompany\\
		\langBegleiter: &\myCompanion \\
		\langAbgabefrist: &\myEndDate
	\end{tabular}
	\newline
	\vspace*{1cm}
	\newline
	\begin{tabularx}{\textwidth}{l@{\extracolsep\fill}r}
	  % Unterschrift des verantwortlichen Ausbilders&\\
	 % (oder des Personalverantwortlichen)&\rule{6cm}{0.3mm}\\
	\end{tabularx}
\end{titlepage}
\newpage
        \or
            \begin{titlepage}
	\begin{center}
		\vspace*{2cm}
		\LARGE\bf\myTopic\\
		\Large\rm\mySubTopic\\
		\vspace*{3cm}
		\bf \langBachelorThesis\\
		\normalsize\rm
		\vspace*{0.5cm}\singlespacing
		\langForDegree \myTypeOfBachelor\\
		\vspace*{0.5cm}\singlespacing
        \ifcase\myNeedForFieldOfStudy
        \langFaculty\\
        \langStudyProgram \myStudyProgram \\
        \or
            \langFaculty\\
        \langStudyProgram \myStudyProgram \\
            \langFieldOfStudy \myFieldOfStudy\\
        \else
        \fi
		\vspace*{0.5cm}\singlespacing
		\langAt\\
		DHBW Ravensburg
		\vfill
	\end{center}
	\begin{tabular}{ll}
		\langAutor: &\myAutor\\
        \langKurs: &\myCourse\\
        \langBetreuer: &\myProf\\
		\langFirma: &\myCompany\\
		\langBegleiter: &\myCompanion \\
		\langAbgabefrist: &\myEndDate
	\end{tabular}
\end{titlepage}
\newpage
        \or
            \begin{titlepage}
	\begin{center}
		\vspace*{1cm}
		\LARGE\bf\myTopic\\
		\Large\rm\mySubTopic\\
		\vspace*{2cm}
		\bf \myTypeOfWork \\
		\vspace*{0.5cm}\singlespacing
		\normalsize\rm
		\ifcase\myNeedForFieldOfStudy
            \langFaculty\\
            \langStudyProgram \myStudyProgram \\
        \or 
            \langFaculty\\
            \langStudyProgram \myStudyProgram \\
            \langFieldOfStudy \myFieldOfStudy\\
        \else
        \fi
		\vspace*{0.5cm}\singlespacing
		\langAt\\
		DHBW Ravensburg
		\vfill
	\end{center}
	\begin{tabular}{ll}
		\ifcase\myNames
            \langAutor:
        \or 
            \langAutoren:
        \else
        \fi
        &\myAutor\\
        \langModul: &\myModule\\
		\langDozent:&\myProf\\
		\langAbgabefrist:&\myEndDate
	\end{tabular}
	
\end{titlepage}
\newpage
        \else
        \fi

    \ifcase\myBlockingNote
        \or
            \thispagestyle{empty} %change

\begin{center}
	\vspace*{1cm}
	\Huge\bf Sperrvermerk\\
	\vspace*{2cm}
	\large\rm
	
	\begin{quotation}
		\hspace*{-1.75em}
		\parbox{0.85\textwidth}{\singlespacing Der Inhalt dieser Arbeit darf weder als Ganzes noch in Auszügen Personen außerhalb des Prüfungsprozesses und des Evaluationsverfahrens zugänglich gemacht werden, sofern keine anderslautende Genehmigung des Dualen Partners vorliegt.}
	\end{quotation}
	\vspace*{0.5cm}
	\begin{quotation}
		\hspace*{-1.5em}
		\parbox{\textwidth}{
			\singlespacing
			\begin{tabularx}{0.83\textwidth}{l@{\extracolsep\fill}l}
				\rule{4cm}{0.3mm}&\rule{4cm}{0.3mm}\\
				\large Ort, Datum&\large Unterschrift
			\end{tabularx}}
	\end{quotation}
	\end{center}
\newpage

        \else
    \fi

	\pagestyle{fancy}
    \setcounter{page}{1} % Reset für Inhaltsverzeichnis (Beginn bei I)
    
    % Header-Definition VOR die Metaseiten schieben
	\fancyhead[L]{\nouppercase{\leftmark}} 
	\include{pages/12_inhaltsverzeichnis}

\ifcase\myAbrevationDirectory

\or
    \newpage
\section*{\langAbkuerzungsverzeichnis}

\markboth{\langAbkuerzungsverzeichnis}{\langAbkuerzungsverzeichnis} 

\addcontentsline{toc}{section}{\langAbkuerzungsverzeichnis}

\begin{acronym} [DHBW]
    \acro{DHBW}{Duale Hochschule Baden-Württemberg}
\end{acronym}

\newpage
\else
\fi
\ifcase\myIllustrationDirectory

\or
    \include{pages/15_abbildungsverzeichnis}
\else
\fi
\ifcase\myTableDirectory

\or
    \include{pages/16_tabellenverzeichnis}
\else
\fi
\ifcase\myListingDirectory

\or
    \include{pages/17_listingsverzeichnis}
\else
\fi
\ifcase\myFormularDirectory

\or
    \newpage

\section*{\langFormelverzeichnis}

\markboth{\langFormelverzeichnis}{\langFormelverzeichnis} 

\addcontentsline{toc}{section}{\langFormelverzeichnis}

% Prefix für die Gleichung (z.B. FV. oder LF.)
\renewcommand{\theequation}{\langFormelPrefix\arabic{equation}} 
\counterwithin{equation}{section}

\newcommand{\dotfillref}[2]{#1 \dotfill #2}

\begin{enumerate}[label={}, leftmargin=0pt, itemindent=*]
    \item \dotfillref{\(E = mc^2\)}{\ref{eq:einstein}}
\end{enumerate}
\else
\fi
\ifcase\mySymbolDirectory

\or   
	\renewcommand{\nomname}{\langSymbolverzeichnis}

\nomenclature{$E$}{Energie (Joule)}
\nomenclature{$m$}{Masse (Kilogramm)}
\nomenclature{$c$}{Lichtgeschwindigkeit im Vakuum (m/s)}

\printnomenclature
\else
\fi


	% Kapitel
	\pagenumbering{arabic}
	%%%%%%%%%%%%%%%%%%%%%%%%%%%%%%%%%%%%%%%%%%%%%%%%%%%%%%%%%%%%%%%%%%%%%%%%%%%%%
%%                                                                         %%
%% \/   \/      Erklärung der Begriffe						       \/   \/ %%
%%                                                                         %%
%%%%%%%%%%%%%%%%%%%%%%%%%%%%%%%%%%%%%%%%%%%%%%%%%%%%%%%%%%%%%%%%%%%%%%%%%%%%%
%% Überschriften und Gliederung				%
\section{erste Gliederungsebene}\label{sec:erste Gliederungseben}		   	% 
\subsection{zweite Gliederungsebene}\label{sec:zweite Gliederungsebene}	   	% 
\subsubsection{dritte Gliederungsebene}	\label{sec:dritte Gliederungsebene}	% 
\subsubsection{dritte Gliederungsebene}	\label{sec:dritte Gliederungsebene_1}	% 
\label{referenzKey} 					   	% Einen Referenzpunkt setzen
%											% 
%% Text Styles								%
\gqq{Anführungszeichen} 				\\ 	% 
\textbf{Fettgedruckt}					\\ 	% 
\textit{Kursiv}							\\ 	% 
\underline{Unterstrichen}				\\ 	% 
\\											% Zeilenumbruch
%											%
%% Text Elements							%
\myboxquote{Dieser Text steht in einer Box} % Für längere Zitate geeignet
\begin{itemize}								% Beginn einer Aufzählung
	\item erster Punkt						% Aufzählungspunkt
	\item Zweiter Punkt						% Aufzählungspunkt
\end{itemize}								% Ende der Aufzählung

\begin{figure}[hbt]							% Beginn einer Grafik
	\centering 								% trim=left bottom right top
	\includegraphics[width=0.75\textwidth]{images/Seeigel_wuhuu.JPG}
	\caption[Bild Beschriftung]{Bild Beschriftung \cite[2]{einstein}}  % zweiter teil nicht im Verzeichnis
	\label{fig:seeigel}
\end{figure}

 
%% BEISPIELE FÜR ZITATIONEN (APA STIL) %%

% 1. Indirektes Zitat (Sinngemäß) -> Nutzt Ihren neuen \vgl Befehl
Das ist eine allgemeine Aussage \vgl{einstein}.

Das steht auf einer Seite \vgl[2]{einstein}.

Das steht auf mehreren Seiten \vgl[2--4]{einstein}.

% 2. Direktes Zitat (Wörtlich) -> Nutzt \parencite (KEIN vgl.)
"Gott würfelt nicht" \parencite[45]{einstein}.

% 3. Im Fließtext (Narrativ) -> Nutzt \textcite
Wie \textcite{einstein} schon 1905 sagte, ist alles relativ.

Nach \textcite[12]{einstein} ist die Lichtgeschwindigkeit konstant.

Referenz auf einen Anhang:\\
\mypageref{bootstrap-book}
\\\\
%											% 
%% Label Referenzierung 					%
(siehe \mypageref{referenzKey})			\\ 	% Referenz auf \label key mit Seitenangabe
Abb. \ref{fig:seeigel}					\\	% Referenz auf \label ohne Seitenangabe
%											%
Nun folgen alle Möglichkeiten, um Abkürzungen im Text einzufügen:\\
\ac{DHBW}\\
\acs{DHBW}\\
\acl{DHBW}\\
\acp{DHBW}\\
%											%
%% Weitere Hinweise
Mehrmals Kompilieren hilft manchmal bei Problemen.\\
Alles außer vorlage.tex und der Ordner aus dem Root-Verzeichnis löschen und mehrmals neu kompilieren löst hartnäckigere Fehler.

\section{Ich bin ein Kapitel} \label{Ich bin ein Kapitel}
Ich bin ein Kapitel
\subsection{Listing Beispiel} \label{Listing Beispiel}
\begin{lstlisting}[language=Python, caption={MeinListing}, label={MeinListing}]
/************Globals & defines**************/

// Pins
const int dpSensor = 2;
const int dpDumpButton = 3;
const int dpGreenLed = 9;
const int dpRedLed = 10;
const int dpValveTank = 7;
const int dpValveDump = 8;

// Global variables
volatile unsigned long ulMicrosWait = 0;
\end{lstlisting}

\subsection{Tabelle Beispiel} \label{Tabelle Beispiel}

\begin{table}[b]
	\centering
	\caption{Beispiel einer Tabelle}
	\begin{tabular}{m{5.5cm} >{\centering\arraybackslash}m{0.7cm} >{\centering\arraybackslash}m{2.3cm} >{\centering\arraybackslash}m{2.3cm} >{\centering\arraybackslash}m{2.3cm}}
		\toprule
		\midrule
		\multicolumn{5}{c}{\textbf{Senica}}\\
		\midrule
		& & Manuell & LM & RM \\
		\cmidrule(l){3-5}
		Netto Produktionszeit $T_{NET}$ & \si{\frac{min}{shift}} & \num{392,0} & \multicolumn{2}{c}{\num{411,8}}\\
		Ausgebrachte Stückzahl pro Jahr $n$ & \si{\frac{stk}{a}} & \num{497954} & \num{686400} & \num{1090164}\\
		\midrule
		\multicolumn{5}{c}{\textbf{Neustadt}}\\
		\midrule
		& & Manuell & LM & RM \\
		\cmidrule(l){3-5}
		Netto Produktionszeit $T_{NET}$ & \si{\frac{min}{shift}} & \num{395,5} & \multicolumn{2}{c}{\num{428,5}}\\
		Ausgebrachte Stückzahl pro Jahr $n$ & \si{\frac{stk}{a}} & \num{517476} & \num{685555} & \num{1088823}\\
		\bottomrule
	\end{tabular}
\end{table}

\subsection{Formel- und Symbol-Beispiel} \label{Mathe Beispiel}
\ac{DHBW}
Hier ein Beispiel für eine Formel:
\begin{equation}
    E = mc^2 \label{eq:einstein}
\end{equation}
Hierbei ist \(E\) die Energie, \(m\) die Masse, und \(c\) die Lichtgeschwindigkeit.

\section{Fazit}
Ich bin ein Fazit.

	% Anhang-Nummerierung
	\renewcommand{\thetable}{\Alph{section}.\arabic{table}}
	\renewcommand{\thefigure}{\Alph{section}.\arabic{figure}}
	\renewcommand{\thelstlisting}{\Alph{section}.\arabic{lstlisting}}

	% Abschluss: Reihenfolge gem. Richtlinie V 2.0 (Anhang VOR Literatur)

    \begin{appendix}
		\include{appendix/index}
	\end{appendix}
    
	\newpage
\printbibliography[heading=bibintoc, title={\langLiteraturverzeichnis}]

    \ifcase\myDecOfIndependence
        \or
            \thispagestyle{empty}

\begin{comment}
    \addcontentsline{toc}{section}{Selbständigkeitserklärung}
\end{comment}

\begin{center}
	\vspace*{2cm}
	\Huge\bf Selbständigkeitserklärung\\
	\vspace*{3cm}
	\large\rm\singlespacing 
	% Ich versichere hiermit, dass ich meine\\ 
    % \ifcase\myType Projektarbeit 
    % \or Bachelorarbeit 
    % \or \myTypeOfWork
    % \else\fi ~mit dem Thema\\
	% \vspace*{2cm}
	% \Large\bf\myTopic\\
	% \Large\rm\mySubTopic\\
	% \vspace*{2cm}
	% \large\rm
	% \singlespacing 
	% selbständig verfasst und keine anderen als die angegebenen\\Quellen und Hilfsmittel benutzt habe. 
	Ich versichere hiermit, dass ich die vorliegende Arbeit selbstständig verfasst und keine anderen als die angegebenen Quellen und Hilfsmittel verwendet habe und diese Arbeit bei keiner anderen Prüfung mit gleichem oder vergleichbarem Inhalt vorgelegt habe und diese bislang nicht veröffentlicht wurde. Des Weiteren versichere ich, dass die eingereichte elektronische Fassung mit der gedruckten Ausfertigung übereinstimmt.
	\vfill
	\begin{tabularx}{\textwidth}{l@{\extracolsep\fill}r}
            \myPlace, \myDate & \\
            \rule{7cm}{0.3mm} & \rule{7.55cm}{0.3mm}
    \end{tabularx}
    \begin{tabularx}{\textwidth}{*{2}{>{\arraybackslash}X}}
        Ort, Datum & Unterschrift 
    \end{tabularx}
\end{center}

            \thispagestyle{empty}
        \else
    \fi
 
\end{document}